% On Ubuntu, use to install
%
%     sudo apt install texlive-full texlive-latex-extra
%
% On Fedora, use to install
%
%     sudo dnf install texlive texlive-latex-extra
%
\documentclass{article}

% Packages
\usepackage[hidelinks]{hyperref}
\usepackage{titlesec}
\usepackage{titling} % Had to install texlive-titling
\usepackage[margin=1.25in]{geometry}

\geometry{
  a4paper,
  top=10mm,
}

% Author Information
\title{R\'esum\'e}
\author{Phillip Dudley}

% Formatting
\titleformat{\section}
  {\bfseries\Large\lowercase}
  {}
  {0em}
  {}[\titlerule]

\titleformat{\subsection}
  {\bfseries}
  {\hspace{-.25in}$\bullet$}
  {0em}
  {}

\titleformat{\subsubsection}[runin]
  {\bfseries}
  {}
  {0em}
  {}[:]

\titlespacing{\subsubsection}
  {.25em}
  {0em}
  {0em}

% Here we're adjusting the maketitle section so that its customized to fit our
% needs. Everything is centered and could probably be changed to be split so it
% looks super pretty for a resume.
\renewcommand{\maketitle}{
  \begin{center}
  {\huge\bfseries\theauthor}

  \vspace{.25em}

  102 Northeast Cresent St
  \hfill
  Phillip.A.Dudley3@gmail.com

  Lee's Summit, Missouri 64086
  \hfill
  1-816-267-9411


  \hfill
  \url{https://filbot3.github.io/FilBot3}
\end{center}
}

\begin{document}

  \maketitle

  \section{Work Experience}

    \subsection{DevOps Engineer II {\hfill} October 2019 - Present}

      \subsubsection{Blue Cross Blue Shield of Kansas City, Mo}

        \begin{scriptsize}
        \begin{itemize}

          \item Create, manage, and deploy to Azure Kubernetes Service
            clusters using CI/CD.

          \item Defined Software Development Life Cycle standards and processes
            for interacting with contracted work forces.

          \item Defined policies for working with Azure DevOps and Software
            Development Life Cycle.

          \item Assisted with the adoption of Azure Cloud Services.

          \item Administered Azure DevOps as well as taught others how to use
            the tool to automate code deployments and infrastructure.

          \item Automated administrative tasks of Azure DevOps.

          \item Work with infrastructure to build foundation skills in
            Infrastructure as Code reducing repeated work as well as Git.

          \item Work with security team to improve secrets management through
            the use of CyberArk and Conjur.

          \item Created and taught the fundamentals of containers and working
            with Docker and Kubernetes as well as other container technologies.

          \item Create pipelines to deploy cloud infrastructure for key
            business initiatives.

        \end{itemize}
        \end{scriptsize}

    \subsection{Senior System Engineer {\hfill} February 2013 - October 2019}

      \subsubsection{Cerner Corporation: Kansas City, Mo}

        \begin{scriptsize}
        \begin{itemize}

          \item Used and Developed Chef Cookbooks and Ansible Playbooks for
            automating deployment of infrastructure and applications, as well
            as to remediate other issues during operations.

          \item Deployed and used Continuous Integration and Delivery
            mechanisms like Jenkins, BuildBot and DroneCI to deliver software
            and manage operations of systems.

          \item Performed network, systems and application monitoring using
            Zabbix Enterprise Monitoring, as well as enhancing existing
            monitoring through the use of script and API calls.

          \item Utilized and deployed Splunk for log aggregation and as a time
            series database for monitoring and historical trending of
            application and system performance.

          \item Managed and setup IBM WebSphere Business Process Manager.
            Scripted deployments of web applications for developers.

          \item Worked with Informatica PowerCenter Enterprise and with the
            developers working on it to save and edit connections via XML
            configurations.

        \end{itemize}
        \end{scriptsize}

  \section{Projects}

    \subsection{Kubernetes}

      \begin{scriptsize}
        \begin{itemize}
          \item Use Terraform and other automation techniques to deploy Kubernetes in Azure, on-premises, and locally.
          \item Create slim, secure container images to deploy applications.
        \end{itemize}
      \end{scriptsize}

    \subsection{Modernizing Infrastructure Practices}

      \begin{scriptsize}
        \begin{itemize}
          \item Use tools like Ansible Core and Ansible Tower to manage infrastructure lifecycles.
          \item Remove the manual intervention of infrastructure staff by teaching how automation works.
        \end{itemize}
      \end{scriptsize}

    \subsection{Continuous Integration \& Continuous Delivery}

      \begin{scriptsize}
        \begin{itemize}
          \item Transition from manual deployments to automated by using Azure DevOps and Jenkins.
          \item Moved from Classic UI Pipelines to YAML code-based Pipelines.
        \end{itemize}
      \end{scriptsize}

  \section{Education}
    \subsection{New Horizons Computer Learning: Azure Architect}

      \begin{scriptsize}
      \begin{itemize}

          \item Azure Architect 300 \hfill March 2020

          \item Azure Architect 301 \hfill April 2020

      \end{itemize}
      \end{scriptsize}

    \subsection{Associate of Arts in Information Technology and Networking}
      \begin{scriptsize}
      University of Phoenix
      \hfill
      Graduated 2014
      \end{scriptsize}

  \section{Languages and Technologies}

    \begin{scriptsize}
    \begin{itemize}

      \item Kubernetes, OpenShift, k3s, k0s, minikube, KinD

      \item Python, Rust, Ruby, Golang, Java, C\#, Git, Bash, Maven

      \item SQL, PostgreSQL, MariaDB, MySQL, Amazon AWS RDS, Hadoop HDFS

      \item Jenkins, BuildBot, DroneCI, RunDeck, Concourse-CI, GoCD, Azure
        DevOps

      \item HashiCorp Vault, Terraform, Packer, Vagrant

      \item Amazon AWS, Microsoft Azure, OpenStack, Docker, VMWare

    \end{itemize}
    \end{scriptsize}

\end{document}
